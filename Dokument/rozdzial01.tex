\chapter{Wstęp}
\section{Wprowadzenie}
Testowy cytat \cite{SQL2}

\section{Układ dokumentu}
W rozdziale pierwszym przedstawiono w zarysie czym jest i czego dotyczy niniejszy dokument (jest to szablon, który można zastosować podczas redagowania pracy dyplomowej inżynierskiej bądź magisterskiej). W rozdziale drugim opisano sposób pracy z szablonem. W kolejnym, trzecim rozdziale, przedstawiono zalecenia dotyczące formatowania dokumentu. Rozdział ten pełni rolę czysto informacyjną (dostarczony szablon zapewnia uzyskanie opisanego tam formatowania). W rozdziale czwartym zwrócono uwagę na redakcję pracy dyplomowej (od strony edytorskiej i merytorycznej). Rozdział piąty poświęcono na uwagi techniczne. Ostatni, szósty rozdział, przeznaczono na kilka słów podsumowania oraz ,,lorem ipsum'' -- wygenerowany tekst, pełniący rolę ,,wypełniacza'', wykorzystany w celach poglądowych (jak dzielić dokument na sekcje). Pracy towarzyszy przykładowy wykaz literatury oraz przykładowe dwa dodatki.