\documentclass[a4paper,onecolumn,oneside,12pt,extrafontsizes]{memoir}
\usepackage[utf8]{inputenc}
\usepackage[T1]{fontenc}
\usepackage[english,polish]{babel}
\usepackage{setspace}
\usepackage{color,calc}
\usepackage{ebgaramond}
\usepackage{tgtermes}
\usepackage{float}
\usepackage{silence}

\renewcommand*\ttdefault{txtt}

\clubpenalty=10000
\widowpenalty=10000
\righthyphenmin=3

\renewcommand{\topfraction}{0.95}
\renewcommand{\bottomfraction}{0.95}
\renewcommand{\textfraction}{0.05}
\renewcommand{\floatpagefraction}{0.35}

\setlength{\headsep}{10pt}
\setlength{\headheight}{13.6pt}
\setlength{\footskip}{\headsep+\headheight}
\setlength{\uppermargin}{\headheight+\headsep+1cm}
\setlength{\textheight}{\paperheight-\uppermargin-\footskip-1.5cm}
\setlength{\textwidth}{\paperwidth-5cm}
\setlength{\spinemargin}{2.5cm}
\setlength{\foremargin}{2.5cm}
\setlength{\marginparsep}{2mm}
\setlength{\marginparwidth}{2.3mm}

\checkandfixthelayout[fixed]

\linespread{1}
\setlength{\parindent}{14.5pt}

\usepackage{multicol}
\usepackage{tabularx}
\usepackage{listings}
\usepackage{xpatch}
\makeatletter
\xpatchcmd\l@lstlisting{1.5em}{0em}{}{}
\makeatother

\lstset{
  basicstyle=\small\ttfamily,
  breaklines=true,
  postbreak=\mbox{\textcolor{red}{$\hookrightarrow$}\space},
  belowskip=.5\baselineskip,
  literate={\_}{{\_\allowbreak}}1
}

\definecolor{lightgray}{rgb}{.9,.9,.9}
\definecolor{darkgray}{rgb}{.4,.4,.4}
\definecolor{purple}{rgb}{0.65, 0.12, 0.82}
\definecolor{javared}{rgb}{0.6,0,0} % for strings
\definecolor{javagreen}{rgb}{0.25,0.5,0.35} % comments
\definecolor{javapurple}{rgb}{0.5,0,0.35} % keywords
\definecolor{javadocblue}{rgb}{0.25,0.35,0.75} % javadoc

\lstdefinelanguage{JavaScript}{
	keywords={typeof, new, true, false, catch, function, return, null, catch, switch, var, if, in, while, do, else, case, break},
	keywordstyle=\color{blue}\bfseries,
	ndkeywords={class, export, boolean, throw, implements, import, this},
	ndkeywordstyle=\color{darkgray}\bfseries,
	identifierstyle=\color{black},
	sensitive=false,
	comment=[l]{//},
	morecomment=[s]{/*}{*/},
	commentstyle=\color{purple}\ttfamily,
	stringstyle=\color{red}\ttfamily,
	morestring=[b]',
	morestring=[b]"
}

\lstdefinestyle{JavaScriptStyle}{
	language=JavaScript,
	commentstyle=\color{javagreen},
	backgroundcolor=,
	extendedchars=true,
	basicstyle=\footnotesize\ttfamily,
	showstringspaces=false,
	showspaces=false,
	numbers=none,%left,
	numberstyle=\footnotesize,
	numbersep=9pt,
	tabsize=2,
	breaklines=true,
	breakatwhitespace=true,
	showtabs=false,
	captionpos=t
}

\definecolor{pblue}{rgb}{0.13,0.13,1}
\definecolor{pgreen}{rgb}{0,0.5,0}
\definecolor{pred}{rgb}{0.9,0,0}
\definecolor{pgrey}{rgb}{0.46,0.45,0.48}
\definecolor{dark-grey}{rgb}{0.4,0.4,0.4}

\newcommand\JSONnumbervaluestyle{\color{blue}}
\newcommand\JSONstringvaluestyle{\color{red}}

\newif\ifcolonfoundonthisline

\makeatletter

\lstdefinestyle{json-style}
{
	showstringspaces    = false,
	keywords            = {false,true},
	alsoletter          = 0123456789.,
	morestring          = [s]{"}{"},
	stringstyle         = \ifcolonfoundonthisline\JSONstringvaluestyle\fi,
	MoreSelectCharTable =%
	\lst@DefSaveDef{`:}\colon@json{\processColon@json},
	basicstyle          = \footnotesize\ttfamily,
	keywordstyle        = \ttfamily\bfseries,
	numbers				= left,
	numberstyle={\footnotesize\ttfamily\color{dark-grey}},
	xleftmargin			= 2em
}

\newcommand\processColon@json{
	\colon@json
	\ifnum\lst@mode=\lst@Pmode
	\global\colonfoundonthislinetrue
	\fi
}

\lst@AddToHook{Output}{
	\ifcolonfoundonthisline
	\ifnum\lst@mode=\lst@Pmode
	\def\lst@thestyle{\JSONnumbervaluestyle}
	\fi
	\fi
	\lsthk@DetectKeywords
}

\lst@AddToHook{EOL}
{\global\colonfoundonthislinefalse}

\makeatother

\usepackage{memlays}
\usepackage{printlen}
\usepackage{enumitem}

\uselengthunit{pt}
\makeatletter
\newcommand{\showFontSize}{\f@size pt} % makro wypisujące wielkość bieżącej czcionki
\makeatother

\usepackage{enumitem}
\setlist{noitemsep,topsep=4pt,parsep=0pt,partopsep=4pt,leftmargin=*}
\setenumerate{labelindent=0pt,itemindent=0pt,leftmargin=!,label=\arabic*.}
\setlistdepth{4}
\setlist[itemize,1]{label=$\bullet$}
\setlist[itemize,2]{label=\normalfont\bfseries\textendash}
\setlist[itemize,3]{label=$\ast$}
\setlist[itemize,4]{label=$\cdot$}
\renewlist{itemize}{itemize}{4}

\makeatletter
\renewenvironment{quote}{
	\begin{list}{}
	{
	\setlength{\leftmargin}{1em}
	\setlength{\topsep}{0pt}%
	\setlength{\partopsep}{0pt}%
	\setlength{\parskip}{0pt}%
	\setlength{\parsep}{0pt}%
	\setlength{\itemsep}{0pt}
	}
	}{
	\end{list}}
\makeatother

\usepackage[pdftex,bookmarks,breaklinks,unicode]{hyperref}
\usepackage{hyperxmp}
\usepackage{ifpdf}

\ifpdf
 \usepackage{datetime2}
 \usepackage[pdftex]{graphicx}
 \DeclareGraphicsExtensions{.pdf,.jpg,.mps,.png}
\pdfcompresslevel=9
\pdfoutput=1

\makeatletter
\AtBeginDocument{
  \hypersetup{
	pdfinfo={
    Title = {\@title},
    Author = {\@author},
    Subject={Praca dyplomowa \ifMaster magisterska\else inżynierska\fi},
    Keywords={\@kvpl},
		Producer={},
	  CreationDate= {},
    ModDate={\pdfcreationdate},
		Creator={pdftex},
	}}
}

\pdftrailerid{}
\pdfsuppressptexinfo15
\makeatother
\else
\usepackage{graphicx}
\DeclareGraphicsExtensions{.eps,.ps,.jpg,.mps,.png}
\fi
\sloppy

\def\UrlBreaks{\do\/\do-\do_}

\setcounter{secnumdepth}{2}
\setcounter{tocdepth}{2}
\setsecnumdepth{subsection}

\makeatletter
\def\@seccntformat#1{\csname the#1\endcsname.\quad}
\def\numberline#1{\hb@xt@\@tempdima{#1\if&#1&\else.\fi\hfil}}
\makeatother

\renewcommand{\chapternumberline}[1]{#1.\quad}
\renewcommand{\cftchapterdotsep}{\cftdotsep}

\makeatletter
\renewcommand*{\insertchapterspace}{%
  \addtocontents{lof}{\protect\addvspace{3pt}}%
  \addtocontents{lot}{\protect\addvspace{3pt}}%
	\addtocontents{toc}{\protect\addvspace{3pt}} %
  \addtocontents{lol}{\protect\addvspace{3pt}}}
\makeatother

\setlength{\cftbeforechapterskip}{0pt}
\renewcommand{\aftertoctitle}{\afterchaptertitle\vspace{-4pt}}

\captionnamefont{\small}
\captiontitlefont{\small}

\newcommand{\listingcaption}[1]
{%
\vspace*{\abovecaptionskip}\small
\refstepcounter{lstlisting}\hfill%
Listing \thelstlisting: #1\hfill%\hfill%
\addcontentsline{lol}{lstlisting}{\protect\numberline{\thelstlisting}#1}
}%

\newcommand{\eng}[1]{(ang.~\emph{#1})}
\newcommand*{\captionsource}[2]{
  \caption[{#1}]{
    #1 \emph{Źródło:} #2
  }%
}

\addto\captionspolish{
\renewcommand{\tablename}{Tab.}
}

\addto\captionspolish{
\renewcommand{\figurename}{Rys.}
}

\addto\captionspolish{
\renewcommand{\lstlistlistingname}{Spis listingów}
}

\addto\captionspolish{
\renewcommand{\bibname}{Literatura}
}

\addto\captionspolish{
\renewcommand{\listfigurename}{Spis rysunków}
}

\addto\captionspolish{
\renewcommand{\listtablename}{Spis tabel}
}

\addto\captionspolish{
\renewcommand\indexname{Indeks rzeczowy}
}

\renewcommand{\abstractnamefont}{\normalfont\Large\bfseries}
\renewcommand{\abstracttextfont}{\normalfont}

\addtopsmarks{headings}{
    \nouppercaseheads
}{
\createmark{chapter}{both}{shownumber}{}{. \space}
\createmark{section}{right}{shownumber}{}{. \space}
}

\makeatletter
\copypagestyle{outer}{headings}
\makeoddhead{outer}{}{}{\small\itshape\rightmark}
\makeevenhead{outer}{\small\itshape\leftmark}{}{}
\makeoddfoot{outer}{\small\@author:~\@titleShort}{}{\small\thepage}
\makeevenfoot{outer}{\small\thepage}{}{\small\@author:~\@title}
\makeheadrule{outer}{\linewidth}{\normalrulethickness}
\makefootrule{outer}{\linewidth}{\normalrulethickness}{2pt}
\makeatother

\copypagestyle{plain}{headings}
\makeoddhead{plain}{}{}{}
\makeevenhead{plain}{}{}{}
\makeevenfoot{plain}{}{}{}
\makeoddfoot{plain}{}{}{}

\copypagestyle{empty}{headings}
\makeoddhead{empty}{}{}{}
\makeevenhead{empty}{}{}{}
\makeevenfoot{empty}{}{}{}
\makeoddfoot{empty}{}{}{}

\newif\ifMaster

\makeatletter
%Uczelnia
\newcommand\uczelnia[1]{\renewcommand\@uczelnia{#1}}
\newcommand\@uczelnia{}
%Wydział
\newcommand\wydzial[1]{\renewcommand\@wydzial{#1}}
\newcommand\@wydzial{}
%Kierunek
\newcommand\kierunek[1]{\renewcommand\@kierunek{#1}}
\newcommand\@kierunek{}
%Specjalność
\newcommand\specjalnosc[1]{\renewcommand\@specjalnosc{#1}}
\newcommand\@specjalnosc{}
%Tytuł po angielsku
\newcommand\titleEN[1]{\renewcommand\@titleEN{#1}}
\newcommand\@titleEN{}
%Tytuł krótki
\newcommand\titleShort[1]{\renewcommand\@titleShort{#1}}
\newcommand\@titleShort{}
%Promotor
\newcommand\promotor[1]{\renewcommand\@promotor{#1}}
\newcommand\@promotor{}
%Słowa kluczowe
\newcommand\kvpl[1]{\renewcommand\@kvpl{#1}}
\newcommand\@kvpl{}
\newcommand\kven[1]{\renewcommand\@kven{#1}}
\newcommand\@kven{}
%Komenda wykorzystywana w streszczeniu
\newcommand\mykeywords{\hspace{\absleftindent}%
\parbox{\linewidth-2.0\absleftindent}{
       \iflanguage{polish}{\textbf{Słowa kluczowe:} \@kvpl}{%
			 \iflanguage{english}{\textbf{Keywords:} \@kven}}{}}
				}

\def\maketitle{
  \pagestyle{empty}
    \fontfamily{\ebgaramond@family}\selectfont

\newlength{\tmpfboxrule}
\setlength{\tmpfboxrule}{\fboxrule}
\setlength{\fboxsep}{2mm}
\setlength{\fboxrule}{0mm}

\setlength{\unitlength}{1mm}
\begin{picture}(0,0)

\put(20,-124){\fbox{
\parbox[c][71mm][c]{104mm}{\centering
{\fontsize{18pt}{20pt}\bfseries\selectfont \@title}\\[5mm]
{\fontsize{18pt}{20pt}\bfseries\selectfont \@titleEN}\\[10mm]
{\fontsize{16pt}{18pt}\selectfont \@author}}
}
}
\end{picture}
\setlength{\fboxrule}{\tmpfboxrule}
{\vskip 9pt\centering
		{\fontsize{20pt}{22pt}\bfseries\selectfont \@uczelnia}\\[5pt]
		{\fontsize{16pt}{18pt}\bfseries\selectfont \@wydzial}\\[1pt]
		  \hrule
	}
{\vskip 24pt\raggedright\fontsize{14pt}{16pt}\selectfont%
\begin{tabular}{@{}ll}
Kierunek: & {\bfseries \@kierunek}\\
Specjalność: & {\bfseries \@specjalnosc}\\
\end{tabular}\\[1.3cm]
}
{\vskip 29pt\centering{\fontsize{24pt}{26pt}\selectfont%
{\fontsize{26pt}{28pt}\selectfont P}RACA {\fontsize{26pt}{24pt}\selectfont D}YPLOMOWA\\[7pt]
\ifMaster \selectfont{\fontsize{26pt}{28pt}\selectfont M}AGISTERSKA\\[2.5cm]%
\else \selectfont{\fontsize{26pt}{28pt}\selectfont I}NŻYNIERSKA\\[2.5cm]\fi
}}
	\vfill
{\centering
    {\fontsize{14pt}{16pt}\selectfont Opiekun pracy}\\[2mm]
	{\fontsize{14pt}{16pt}\bfseries\selectfont \@promotor}\\[10mm]
}
\vspace{4cm}\noindent
{\fontsize{12pt}{14pt}\selectfont Słowa kluczowe: \@kvpl}
\vspace{1.3cm}
\hrule\vspace*{0.3cm}
{\centering
{\fontsize{14pt}{16pt}\selectfont \@date}\\[0cm]
}
\normalfont
 \cleardoublepage
}
\makeatother

\title{Aplikacja webowa do optymalizacji zleceń w transporcie towarów} % INFO: tytuł pracy w języku polskim
\titleShort{Aplikacja webowa do optymalizacji zleceń w transporcie towarów}  % INFO: krótki tytuł pracy (do zamieszczenia w stopce, sklejony z imieniem i nazwiskiem autora nie powinien zająć więcej niż jedną linijkę)
\titleEN{Web application for optimizing orders in the transport of goods} % INFO: tytuł pracy w języku angielskim
\author{Krystian Tomczyk}  % INFO: imię i nazwisko autora
\uczelnia{Politechnika Wrocławska} % INFO: nazwa uczelni
\wydzial{Wydział Informatyki i Telekomunikacji} % INFO: nazwa wydziału
\kierunek{Informatyka techniczna (ITE)} % IFO: nazwa kierunku
\specjalnosc{Inżynieria systemów informatycznych (INS)} % INFO: nazwa specjalności
\promotor{dr. inż. Paweł Rogaliński} % INFO: dane promotora
\kvpl{aplikacja, transport, web} % INFO: słowa kluczowe po polsku
\kven{application, transport, web} % INFO: słowa kluczowe po angielsku
\date{WROCŁAW, 2024} % INFO: miejscowość, rok złożenia pracy dyplomowej

\begin{document}

\pdfbookmark[0]{Tytuł}{Tytul.1}
\maketitle

\pdfbookmark[0]{Streszczenie}{streszczenie.1}
\begin{abstract}
Tu będzie streszczenie po Polsku
\end{abstract}
\mykeywords
{
\selectlanguage{english}
\begin{abstract}
Tu będzie streszczenie po angielsku
\end{abstract}
\mykeywords
}

\pagestyle{outer}
\clearpage

\pdfbookmark[0]{Spis treści}{spisTresci.1}
\tableofcontents*
\clearpage

\pdfbookmark[0]{Spis rysunków}{spisRysunkow.1}
\listoffigures*
\clearpage

% \pdfbookmark[0]{Spis tabel}{spisTabel.1}
% \listoftables*
% \clearpage

% \pdfbookmark[0]{Spis listingów}{spisListingow.1}
% \lstlistoflistings*
% \clearpage

\chapterstyle{default}
\chapter{Wstęp}
\texttt{Stan na dzień: \today}

Transport towarów odgrywa kluczową rolę w globalnej gospodarce, łącząc producentów i konsumentów na całym świecie. Efektywność transportu ma bezpośredni wpływ na koszty operacyjne firm oraz na ceny finalnych produktów. W zależności od specyfiki przewożonych towarów oraz potrzeb zleceniodawców, transport może przyjmować dwie formy:

\label{sec:przewoz_regularny}
Transport regularny, inaczej łańcuch dostaw, to przewóz towarów, który odbywa się według ustalonego harmonogramu i stałych tras. Charakteryzuje się regularnością kursów, co oznacza, że pojazdy wykonują swoje trasy w określonych, z góry ustalonych terminach. Przykładami transportu regularnego są linie autobusowe, kolejowe czy lotnicze, które działają według stałego rozkładu jazdy.

\label{sec:transport_okazjonalny}
Transport okazjonalny to przewóz towarów, który nie spełnia definicji przewozu regularnego. Oznacza to, że odbywa się on bez ustalonego z góry rozkładu jazdy. Pojazdy wykonują swoje trasy w zależności od zapotrzebowania klientów, najczęściej jest to usługa jednorazowa. Sam przewóz zaś zlecany jest na potrzebę klienta, nie musi on jednak określnać dokładnego terminu odbycia trasy, ani przez kogo ma on być zrealizowany.

Podczas swoich tras przewoźnicy czasami są zmuszeni do przebycia części drogi bez żadnego załadunku. Powoduje to, że trasy nie są w pełni zoptymalizowane względem kosztów, jakie niesie za sobą pokonywana trasa. Możliwe jest jednak zredukowanie występowania takich sytuacji poprzez odpowiednie powiązanie przewoźników i osób zlecających transport. Zleceniodawca, który nie potrzebuje dostawy towaru w konkretnej dacie, mógłby wtedy zlecić transport z nieokreślonym dokładnie terminem dotarcia towaru, w zamian za niższe ceny przewozowe. Przykład: dyrektor szkoły, w czasie wakacji, zamówił dużych rozmiarów tablicę interaktywną, która nie zmieściłaby się w standardowym samochodzie osobowym. Z racji, że zamówienie zostało złożone w czasie, gdy dzieci nie chodzą do szkoły, nie zależy mu na dokładnym terminie dostawy. Może on w takim przypadku zlecić dostawę tablicy w formie transportu okazjonalnego, z mniejszymi kosztami transportu. Przewoźnik mógłby zabrać towar i zawieźć go na miejsce docelowe, gdy akurat odbywałby trasę bez załadunku i kierował się w tym przybliżonym kierunku. Taka sytuacja jest korzystna dla obu stron, przewoźnik może odbywać trase bardziej efektywnie, dzięki nie marnowaniu zasobów na puste przebiegi. Zleceniodawca natomiast, generują mniejsze koszty, związane z brakiem konieczności korzystania z droższych określonych terminowo usług transpotowych.

Połączenie między osobami zlecającymi usługi transportowe - zleceniodawcami, a osobami oferującymi przewóz towaru - przewoźnikami, może odbywać się za pomocą serwisu oferującego dodawanie publicznych ogłoszeń. Ogłoszenia dzielić się będą na dwie kategorie, ogłoszenie z planowaną trasą oraz zlecenie z wymaganym towarem do przewiezienia do punktu docelowego.

\label{sec:cele}
\section{Cel projektu}
Celem projektu jest stworzenie aplikacji webowej, pod tytułem \texttt{CargoLink}. Serwis ten pozwalać będzie na dodawanie ogłoszeń oraz zleceń dotyczących transportów okazjonalnych. Aplikacja przyczni się do zoptymalizowania procesów logistycznych, eliminując nieefektywne wykorzystanie zasobów transportowych. Dodatkowo pozwoli ona przewoźnikom i zleceniodawcom, na łatwą i szybką komunikacje między sobą.
Główne cele projektu:
\begin{enumerate}
    \item \textbf{Możliwość umieszczania ogłoszeń i zleceń}, aplikacja pozwalać ma na dodawanie ogłoszeń o trasach, planowanych przez przewoźników oraz zleceń transportowych na konkretny towar przez zleceniodawców
    \item \textbf{Ułatwienie szukania odpowiednich ogłoszeń i zleceń}, poprzez system powiązania zleceń transportowych do planowanych tras przewoźników oraz ogłoszeń o planowanej trasie do zleceń zamieszczonych w serwisie, aplikacja skróci czas potrzebny na znalezienie odpowiednich ofert.
    \item \textbf{Komunikacja między przewoźnikami i zleceniodawcami}, aplikacja umożliwi szybką komunikację między użytkownikami serwisu poprzez czat tekstowy.
    \item \textbf{Redukcja kosztów transportu}, dzięki lepszemu dopasowaniu potencjalnych przewoźników i zleceniodawców, aplikacja pozwoli na obniżenie kosztów transportu zarówno dla zleceniodawców, jak i przewoźników.
\end{enumerate}
Nowoczesna aplikacja transportowa przyczyni się do efektywnego zrealizowania tych celów, co wspomoże rynek zleceń transportowych, przynosząc korzyści zarówno dla zleceniodawców, jak i przewoźników.

\section{Wymagania aplikacji}
Analizując cele wymienione w podrozdziale \hyperref[sec:cele]{Cel projektu} oraz opis samej aplikacji, można wywnioskować, że do efektywnego działania serwisu, będą musiały zostać zrealizowane następujące wymagania funkcjonalne:
\begin{enumerate}
    \item Uwierzytelnianie: aplikacja będzie wykorzystywała system rejestracji oraz logowania.
    \item Dodawanie zleceń transportowych: zleceniodawcy powinni mieć możliwość zlecenia przewozu towaru, serwis pomagał będzie znaleźć odpowiedniego przewoźnika, poprzez udostępnienie możliwości dodania ogłoszenia zlecenia. W ogłoszeniu zlecenia znajdować się będzie:
    \begin{itemize}
        \item wymagana trasa (miejsce startu oraz miejsce docelowe),
        \item termin dostarczenia (przedział dat),
        \item waga towarów do przewiezenia,
        \item wymiary przewożonych dóbr,
        \item kategoria każdego z towarów,
        \item informacja o specjalnych warunków podczas transportu (niewymagane),
        \item wynagrodzenie (z zaznaczeniem czy kwota podlega negocjacji),
        \item imienia i nazwiska zleceniodawcy bądź nazwy firmy, która przewoźnik reprezentuje,
        \item ocena zleceniodawcy, wraz z komentarzami,
        \item opis zlecenia (niewymagane),
    \end{itemize}
    \item Dodawanie ogłoszeń o planowanej trasie: system powinien pozwalać przewoźnikom, na dodawanie publicznych informacji o planowanych przez siebie trasach. Ogłoszenie będzie składało się z:
    \begin{itemize}
        \item planowanej trasy (miejsce startu oraz miejsce docelowe),
        \item daty planowanej trasy,
        \item dostępnego miejsca w pojeździe (wymiary liczone w europaletach),
        \item maksymalnej wagi towaru,
        \item danych kontaktowych,
        \item imienia i nazwiska przewoźnika bądź nazwy firmy, która przewoźnik reprezentuje,
        \item oceny przewoźnika, wraz z komentarzami,
        \item opisu ogłoszenia (niewymagane),
    \end{itemize}
    \item System rekomendacji ogłoszeń: podczas wprowadzania danych o trasie, użytkownik będzie informowany o sugerowanych zleceniach dodanych przez innych użytkowników (np. gdy zlecenie dotyczy trasy, która przewoźnik planuje się poruszać). Analogicznie gdy zleceniodawca zamierza dodać ogłoszenie zlecenia, zostanie on poimformowany o proponowanych ogłoszeniach przewoźników.
    \item Umożliwienie konaktu między użytkownikami: jednym z założeń projektowych jest dodanie czatu tekstowego umożliwiającego korespondencje między zleceniodawcami, a przewoźnikami, bezpośrednio w aplikacji. Ma on pełnić rolę komunikacji na wzór tej  oferowanej przez tradycyjną poczte elektroniczną.
    \item Generowanie umowy: przewoźnik i zleceniodawca, po negocjacji warunków umowy, otrzymają wygenerowany przez serwis dokument finalizujący transakcje.
    \item Weryfikacja dodawanych ogłoszeń: zanim ogłoszenie wyświetlać się będzie dla wszystkich użytkowników, wymagana będzie akceptacja jednego z moderatorów serwisu.
    \item Graficzne przedstawienie trasy: w ogłoszeniach dodanych przez użytkowników, wyświetlana będzie mapa z zaznaczoną trasą. Ułatwi to użytkownikom zobrazowanie planowanego kursu.
    \item Regulanim: podczas rejestrowania się do serwisu, użytkownik musi zaakceptować regulamin korzystania z aplikacji. 
\end{enumerate}
Aplikacja powinna być niezawodna i przyjazna do użytkowania dla wszystkich. Do komfortowego korzystania z serwisu przez użytkowników, niezbędna będzie realizacja następujących wymagań niefunkcjonalnych
\begin{enumerate}
    \item Innowacyjność: wykorzystanie nowoczesnych technologii, takich jak \texttt{TypeScript}, \texttt{Next.js}, \texttt{Tailwind CSS}, \texttt{Node.js} oraz \texttt{PostgreSQL}, zapewni wysoką wydajność, skalowalność i bezpieczeństwo aplikacji.
    \item Intuicyjny interfejs użytkownika: Aplikacja będzie posiadać prosty i intuicyjny interfejs użytkownika, który umożliwi łatwą obsługę zarówno dla zleceniodawców, jak i przewoźników.
    \item Dostępność na różnych urządzeniach: Aplikacja będzie responsywna i dostosowana do różnych urządzeń, takich jak komputery, tablety i smartfony, co zapewni wygodę użytkowania w dowolnym miejscu i czasie.
    \item Wielojęzyczność: użytkownicy korzystający z aplikacji, będą mieli możliwość wyboru jednego z trzech przewidzianych języków: polski, angielski oraz niemiecki. Dodatkowo użytkownicy podczas rejestracji będą mieli możliwość wybrania języków, którymi się posługują. Informacje te będą zamieszczone na profilu użytkownika.
\end{enumerate}
\chapter{Wstęp}

\section{Transport okazjonalny}
\label{sec:przewoz_regularny}
Przewóz regularny to przewóz towarów, który odbywa się według ustalonego harmonogramu i stałych tras. Charakteryzuje się regularnością kursów, co oznacza, że pojazdy wykonują swoje trasy w określonych, z góry ustalonych terminach. Przykładami transportu regularnego są linie autobusowe, kolejowe czy lotnicze, które działają według stałego rozkładu jazdy.

\label{sec:transport_okazjonalny}
Transport okazjonalny to przewóz towarów, który nie spełnia definicji przewozu regularnego. Oznacza to, że odbywa się on bez ustalonego z góry rozkładu jazdy i może dotyczyć zarówno tras krajowych, jak i międzynarodowych. Pojazdy wykonują swoje trasy w zależności od zapotrzebowania klientów. Sam przewóz zaś zlecany jest na potrzebę klienta, nie określna on jednak, dokładnego terminu odbycia trasy, ani przez kogo ma on być zrealizowany.

Transport odgrywa kluczową rolę w gospodarce, dlatego ważne jest jak najefektywniejsze wykorzystanie zasobów. Podczas swoich tras przewoźnicy czasami są zmuszeni do powrotu z miejsca docelowego bez żadnego załadunku. Powoduje to, że trasy nie są w pełni zoptymalizowane względem kosztów, jakie niesie za sobą pokonywana trasa. Możliwe jest jednak zredukowanie występowania takich sytuacji poprzez odpowiednie powiązanie przewoźników i osób zlecających transport. Zleceniodawca, który nie potrzebuje dostawy towaru w konkretnej dacie, mógłby wtedy zlecić transport z nieokreślonym dokładnie terminem dotarcia towaru, w zamian za niższe ceny przewozowe. Przykład: dyrektor szkoły, w czasie wakacji, zamówił dużych rozmiarów tablicę interaktywną, która nie zmieściłaby się w standardowym samochodzie osobowym. Z racji, że zamówienie zostało złożone w czasie, gdy dzieci nie chodzą do szkoły, nie zależy mu na dokładnej dacie dostawy. Może on w takim przypadku zlecić dostawę tablicy w formie transportu okazjonalnego, z mniejszymi kosztami transportu. Przewoźnik przeglądający takie zlecenia mógłby zabrać towar i zawieźć go na miejsce docelowe, gdy akurat odbywałby trasę bez załadunku i kierował się w tym przybliżonym kierunku. 

Połączenie między zleceniodawcami, a przewoźnikami, mogłoby się odbywać za pomocą serwisu oferującego dodawanie publicznych ogłoszeń przez obie strony. Prowadziłoby to do sytuacji, w której obie strony korzystają. Przewoźnicy, ponieważ zmniejszyło by to występowanie tras bez załadunku, tym samym zwiększając zyski z tras powrotnych. Zleceniodawcy, natomiast generują mniejsze koszty, związane z brakiem konieczności korzystania z droższych określonych terminowo usług transpotowych. 

Jednak niesie to za sobą zagrożenia co do wiarygodności kwalifikacji przewoźników, gdy jakaś firma decyduje się na regularne usługi transportowe swoich towarów, zazwyczaj ma pełną świadomość o kwalifikacjach do przewozu swoich przewoźników. Transporty okazjonalne, jednakże charakteryzują się tym, że są realizowane przez różne osoby, więc potrzebna jest weryfikacja prawdziwości ich kwalifikacji.

\label{sec:cele}
\section{Cel projektu}
Celem projektu jest stworzenie aplikacji webowej pozwalającej na dodawanie ogłoszeń transportów okazjonalnych, która zoptymalizuje procesy logistyczne i wyeliminuje nieefektywne wykorzystanie zasobów transportowych. Aplikacja umożliwi użytkownikom łatwe i szybkie znalezienie odpowiedniego przewoźnika lub zleceniodawcy, co spowoduje redukcje pustych przebiegów i tym samym kosztów transportu.
Główne cele projektu:
\begin{enumerate}
    \item Ułatwienie szukania odpowiednich ofert: Poprzez automatyzację procesu wyszukiwania i dopasowywania zleceń transportowych, aplikacja usprawni komunikację między zleceniodawcami a przewoźnikami, skracając czas potrzebny na znalezienie odpowiedniego transportu.
    \item Eliminacja pustych przebiegów: Aplikacja umożliwi przewoźnikom znalezienie ładunków na trasach powrotnych, co zmniejszy liczbę pustych przebiegów, przyczyniając się tym do wyższej efektywności wykorzystania zasobów.
    \item Redukcja kosztów transportu: Dzięki lepszemu dopasowaniu potencjalnych kontrahentów dla usług transportowych, aplikacja pozwoli na obniżenie kosztów transportu zarówno dla zleceniodawców, jak i przewoźników.
    \item Poprawa bezpieczeństwa i jakości usług: Aplikacja umożliwi weryfikację kwalifikacji przewoźników oraz ocenę jakości świadczonych usług, co przyczyni się do zwiększenia bezpieczeństwa i zadowolenia klientów.
\end{enumerate}
Nowoczesna aplikacja transportowa przyczyni się do efektywnego zrealizowania tych celów, co wspomoże rynek zleceń transportowych, przynosząc korzyści zarówno dla zleceniodawców, jak i przewoźników.

\section{Wymagania aplikacji}
Analizując cele wymienione w podrozdziale \hyperref[sec:cele]{1.1} oraz opis samej aplikacji, można wywnioskować, że do efektywnego działania serwisu, będą musiały zostać zrealizowane następujące wymagania funkcjonalne:
\begin{enumerate}
    \item Dodawanie ogłoszeń o planowanej trasie: system powinien pozwalać przewoźnikom, na dodawanie publicznych informacji o planowanych przez siebie trasach. Ogłoszenie będzie składało się z:
    \begin{itemize}
        \item planowanej trasy (punkty A i B na mapie),
        \item daty planowanej trasy,
        \item dostępnego miejsca w pojeździe (wymiary liczone w europaletach),
        \item maksymalnej wagi towaru,
        \item listy kwalifikacji autora ogłoszenia,
        \item danych kontaktowych,
        \item pseudnonimu autora,
        \item oceny przewoźnika, wraz z komentarzami,
        \item opisu ogłoszenia (niewymagane),
    \end{itemize}
    \item Dodawanie zleceń transportowych: zleceniodawcy powinni mieć możliwość zlecenia przewozu towaru, serwis pomagał będzie znaleźć odpowiedniego przewoźnika, poprzez udostępnienie możliwości dodania ogłoszenia zlecenia. W ogłoszeniu zlecenia znajdować się będzie:
    \begin{itemize}
        \item wymagana trasa (punkty A i B na mapie),
        \item termin dostarczenia (od do),
        \item informacja o towarze do przewiezenia (takie jak rodzaj, waga, wymiary czy specjalne warunki),
        \item wynagrodzenie (z zaznaczeniem czy kwota podlega negocjacji),
        \item wymagane kwalifikacje przewoźnika,
        \item pseudonim autora,
        \item ocena zleceniodawcy, wraz z komentarzami,
        \item opis zlecenia (niewymagane),
    \end{itemize}
    \item Uwierzytelnianie: aplikacja będzie wykorzystywała system rejestracji oraz logowania. W serwisie weryfikacją użytkowników zajmować się będzie moderator.
    \item System powiązań: podczas wprowadzania danych o trasie, użytkownik będzie informowany o dostępnych zleceniach dodanych przez innych użytkowników.
    \item Umożliwienie konaktu między użytkownikami: jednym z założeń projektowych jest dodanie czatu tekstowego umożliwiającego korespondencje między zleceniodawcami, a przewoźnikami, bezpośrednio w aplikacji. Ma on pełnić rolę komunikacji na wzór tej  oferowanej przez tradycyjną poczte elektroniczną.
    \item Podpisywanie umowy: użytkownicy, po negocjacji warunków umowy, otrzymają wygenerowany przez serwis dokument finalizujący transakcje.
    \item System potwierdzania kwalifikacji: aplikacja, chcąc zachować maksymalne bezpieczeństwo użytkowników, będzie wymagać potwierdzania kwalifikacji przewoźników. Aby kwalifikacja mogła się wyświetlać na profilu przewoźnika, potrzebna jest weryfikacja uprawnień przeprowadzana przez moderatorów serwisu.
    \item Weryfikacja dodawanych ogłoszeń: zanim ogłoszenie wyświetlać się będzie dla wszystkich użytkowników, wymagana będzie akceptacja jednego z moderatorów serwisu.
    \item Graficzne przedstawienie trasy: w ogłoszeniach dodanych przez użytkowników, wyświetlana będzie mapa z zaznaczoną trasą. Ułatwi to użytkownikom zobrazowanie planowanego kursu.
    \item Finansowanie aplikacji: serwis musi generować przychód, który pokryje koszty utrzymania osób na stanowisku moderatorów i administratorów. Zrealizowane to będzie poprzez pobieranie prowizji wynoszącej 5% od każdej podpisanej w aplikacji umowy.
\end{enumerate}
Aplikacja powinna być niezawodna i przyjazna do użytkowania dla wszystkich. Do komfortowego korzystania z serwisu przez użytkowników, niezbędna będzie realizacja następujących wymagań niefunkcjonalnych
\begin{enumerate}
    \item Innowacyjność: wykorzystanie nowoczesnych technologii, takich jak \texttt{TypeScript}, \texttt{Next.js}, \texttt{Tailwind CSS}, \texttt{Node.js} oraz \texttt{PostgreSQL}, zapewni wysoką wydajność, skalowalność i bezpieczeństwo aplikacji.
    \item Intuicyjny interfejs użytkownika: Aplikacja będzie posiadać prosty i intuicyjny interfejs użytkownika, który umożliwi łatwą obsługę zarówno dla zleceniodawców, jak i przewoźników.
    \item Dostępność na różnych urządzeniach: Aplikacja będzie responsywna i dostosowana do różnych urządzeń, takich jak komputery, tablety i smartfony, co zapewni wygodę użytkowania w dowolnym miejscu i czasie.
    \item Wielojęzyczność: użytkownicy korzystający z aplikacji, będą mieli możliwość wyboru jednego z trzech przewidzianych języków: polski, angielski oraz niemiecki. Co przełoży się na międzynarodowy aspekt aplikacji.
    \item Regulanim: podczas rejestrowania się do serwisu, użytkownik musi zaakceptować regulamin korzystania z aplikacji. Zabezpieczy to aplikacje od strony prawnej.
\end{enumerate}

\section{Przypadki użycia}
Biorąc pod uwagę założenia opisane w poprzednich podrozdziałach, zaprojektowany został diagram przypadków użycia aplikacji. Idnetyfikacja aktorów:
\begin{enumerate}
\item Użytkownik - ogólny użytkownik systemu. Reprezentuje dowolną osobę korzystającą z serwisu, jego przypadki użycia będą dziedziczone przez pozostałych aktorów systemu.
\begin{figure}[H]
	\centering
		\includegraphics[width=0.6\linewidth]{rozdzial1/dziedziczenie.png}
	\caption{Graficzne ukazanie dziedziczenia możliwości aktorów}
	\label{Rys. fig:Graficzne ukazanie dziedziczenia możliwości aktorów}
\end{figure}
\item Przewoźnik - aktor odpowiedzialny za transport towarów. Przewoźnik może aktualizować swoje kwalifikacje, przeglądać dostępne zlecenia, dodawać ogłoszenia o planowanych trasach, komunikować się z autorami ogłoszeń, przyjmować zlecenia oraz oceniać i komentować kontrahentów.
\item Zleceniodawca - użytkownik systemu, który zleca transport towarów. Może dodawać nowe zlecenia transportowe, podobnie jak przewoźnik, może również przeglądać ogłoszenia przewoźników oraz komunikować się z autorami ogłoszeń.
\item Moderator - osoba odpowiedzialna za zarządzanie systemem. Moderator potwierdza kwalifikacje przewoźników, zatwierdza lub usuwa nowe ogłoszenia i zlecenia oraz blokuje konta użytkowników.
\item Administrator - użytkownik umiejscowiony najwyżej w hierarchii systemu. Może on wykonywać wszystko co moderator, lecz ma również możliwość dodawania nowych moderatorów lub usuwania obecnych.
\end{enumerate}

\begin{figure}[H]
	\centering
		\includegraphics[width=0.9\linewidth]{rozdzial1/glowne_zalozenia.png}
	\caption{Diagram głównych funkcjonalności aplikacji}
	\label{Rys. fig:Diagram głównych funkcjonalności aplikacji}
\end{figure}

Na powyższym obrazku przedstawiony został diagram przypadków użycia dla przewoźnika oraz zleceniodawcy.

\begin{figure}[H]
	\centering
		\includegraphics[width=0.9\linewidth]{rozdzial1/ogolny_schemat.png}
	\caption{Główne założenia projektowe od strony zarządzania serwisem}
	\label{Rys. fig:Główne założenia projektowe od strony zarządzania serwisem}
\end{figure}

Diagram ukazujący główne założenia systemu od strony zarządzania serwisem, w tym możliwości użytkownika, moderatora oraz administratora.

\bibliographystyle{plabbrv}
\setlength{\bibitemsep}{2pt}
\addtocontents{toc}{\addvspace{2pt}}
\bibliography{bibliografia}
\appendix

\chapter{Instrukcja wdrożeniowa}
Aplikacja jest konteneryzowana przy użyciu Docker, co zapewnia kompatybilność międzyplatformową. Baza danych jest hostowana na darmowym serwerze \texttt{Neon}, co upraszcza konfigurację.
\begin{itemize}
    \item Zainstalowane oprogramowanie \texttt{Docker}
    \item Dostęp do internetu
    \item Uprawnienia administratora
\end{itemize}
W celu wdrożenia aplikacji na środowisko produkcyjne należy wykonać następujące kroki:
\begin{enumerate}
    \item Sklonować repozytorium z GitHub.
    \item Otworzyć terminal w katalogu projektu.
    \item Zbudować obraz aplikacji za pomocą komendy \texttt{docker build -t cargolink .}.
    \item Uruchomić kontener z aplikacją za pomocą komendy \texttt{docker run -p 80:3000 cargolink}.
\end{enumerate}

Po wykonaniu powyższych kroków aplikacja będzie dostępna pod adresem \texttt{http://localhost:80}. Port na którym działa aplikacja można zmienić w pliku \texttt{Dockerfile} w linii 4. Jednak port 80 jest domyślnym portem dla protokołu HTTP, więc zaleca się pozostawienie go bez zmian. Jeżeli serwer, na którym wykonane zostały powyższe kroki, jest dostępny z zewnątrz, aplikacja będzie dostępna pod adresem \texttt{http://<adres\_serwera>:80}.

\end{document}
