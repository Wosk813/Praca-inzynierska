\pdfbookmark[0]{Streszczenie}{streszczenie.1}
\begin{abstract}
Niniejsza praca inżynierska dotyczy projektu aplikacji webowej CargoLink, zaprojektowanej w celu optymalizacji procesów logistycznych w transporcie towarów. Głównym celem systemu jest łączenie zleceniodawców i przewoźników, minimalizowanie pustych przebiegów oraz redukcja kosztów transportu. Aplikacja oferuje funkcje takie jak: dodawanie zleceń transportowych, publikowanie ogłoszeń o planowanych trasach, rekomendacje dopasowanych ofert oraz komunikację użytkowników poprzez wbudowany czat. System wspiera generowanie szablonów umów oraz umożliwia ocenę użytkowników po zakończeniu współpracy.

Pod względem technicznym aplikacja wykorzystuje nowoczesne technologie, takie jak TypeScript, Next.js, Tailwind CSS i PostgreSQL, zapewniające wysoką wydajność i bezpieczeństwo. W bazie danych zastosowano UUID jako klucze główne oraz algorytm bcrypt do szyfrowania haseł. Interfejs użytkownika zaprojektowano w programie Figma z uwzględnieniem responsywności oraz intuicyjnej obsługi.

Dokumentacja opisuje szczegółowo przypadki użycia systemu dla różnych aktorów oraz architekturę warstwową aplikacji. Przedstawiono również proces testowania systemu, w tym testy jednostkowe, end-to-end oraz wydajnościowe. Na zakończenie wskazano potencjalne kierunki rozwoju projektu, takie jak wprowadzenie nowych funkcjonalności i zwiększenie skalowalności systemu.
\end{abstract}
\mykeywords
{
\selectlanguage{english}
\begin{abstract}
This engineering thesis concerns the design of CargoLink web application, developed to optimize logistics processes in cargo transportation. The main goal of the system is to connect shippers with carriers, minimize empty runs, and reduce transportation costs. The application offers features such as: adding transport orders, publishing announcements about planned routes, recommending matched offers, and enabling user communication through a built-in chat. The system supports generating contract templates and allows users to rate each other after completed cooperation.

From a technical perspective, the application utilizes modern technologies such as TypeScript, Next.js, Tailwind CSS, and PostgreSQL, ensuring high performance and security. The database implements UUID as primary keys and uses the bcrypt algorithm for password encryption. The user interface was designed in Figma with consideration for responsiveness and intuitive operation.

The documentation thoroughly describes use cases for various actors and the layered architecture of the application. The system testing process is also presented, including unit tests, end-to-end tests, and performance tests. Finally, potential directions for project development are indicated, such as introducing new functionalities and increasing system scalability.
\end{abstract}
\mykeywords
}
