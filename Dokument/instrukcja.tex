\chapter{Instrukcja wdrożeniowa}
Aplikacja jest konteneryzowana przy użyciu Docker, co zapewnia kompatybilność międzyplatformową. Baza danych jest hostowana na darmowym serwerze \texttt{Neon}, co upraszcza konfigurację. \\
Wymagania:
\begin{itemize}
    \item Zainstalowane oprogramowanie \texttt{Docker}
    \item Dostęp do internetu
    \item Uprawnienia administratora
\end{itemize}
W celu wdrożenia aplikacji na środowisko produkcyjne należy wykonać następujące kroki:
\begin{enumerate}
    \item Sklonować repozytorium z GitHub.
    \item Otworzyć terminal w katalogu projektu.
    \item Zbudować obraz aplikacji za pomocą komendy \texttt{docker build -t cargolink .}.
    \item Uruchomić kontener z aplikacją za pomocą komendy \texttt{docker run -p 80:3000 cargolink}.
\end{enumerate}

Po wykonaniu powyższych kroków aplikacja będzie dostępna pod adresem \texttt{http://localhost:80}. Port na którym działa aplikacja można zmienić w modyfikując argument w \texttt{-p} w komendzie \texttt{docker run -p 80:3000 cargolink}. Port 80 jest domyślnym portem dla protokołu HTTP, więc zaleca się pozostawienie go bez zmian. Jeżeli serwer, na którym wykonane zostały powyższe kroki, jest dostępny z zewnątrz, aplikacja będzie dostępna pod adresem \texttt{http://<adres\_serwera>:80}.