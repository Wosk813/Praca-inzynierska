\chapter{Wstęp}

Transport towarów odgrywa kluczową rolę w globalnej gospodarce, łącząc producentów i konsumentów na całym świecie. Efektywność transportu ma bezpośredni wpływ na koszty operacyjne firm oraz na ceny finalnych produktów \cite{MurphyWoodLogistyka}. W zależności od specyfiki przewożonych towarów oraz potrzeb zleceniodawców, transport może przyjmować dwie następujące formy:

\label{sec:przewoz_regularny}
Transport regularny, inaczej łańcuch dostaw, to przewóz towarów, który odbywa się według ustalonego harmonogramu i stałych tras. Charakteryzuje się regularnością kursów, co oznacza, że pojazdy wykonują swoje trasy w określonych, z góry ustalonych terminach. Przykładami transportu regularnego są linie autobusowe, kolejowe czy lotnicze, które działają według stałego rozkładu jazdy.

\label{sec:transport_okazjonalny}
Transport okazjonalny to przewóz towarów, który odbywa się bez ustalonego z góry rozkładu jazdy. Pojazdy wykonują swoje trasy w zależności od zapotrzebowania klientów, najczęściej jest to usługa jednorazowa. Sam przewóz zaś zlecany jest na potrzebę klienta, nie musi on jednak określać dokładnego terminu odbycia trasy, ani przez kogo ma on być zrealizowany.

Podczas swoich tras przewoźnicy czasami są zmuszeni do przebycia części drogi bez żadnego ładunku. Powoduje to, że przewozy nie są w pełni zoptymalizowane względem kosztów, jakie niesie za sobą pokonywana trasa. Możliwe jest jednak zredukowanie występowania takich sytuacji poprzez odpowiednie powiązanie przewoźników i osób zlecających transport okazjonalny. Zleceniodawca, który nie potrzebuje dostawy towaru w konkretnej dacie, mógłby wtedy zlecić transport z nieokreślonym dokładnie terminem dotarcia towaru, w zamian za niższe ceny przewozowe. Przykład: dyrektor szkoły, w czasie wakacji, zamówił dużych rozmiarów tablicę interaktywną, która nie zmieściłaby się w standardowym samochodzie osobowym. Z racji, że zamówienie zostało złożone w czasie, gdy dzieci nie chodzą do szkoły, nie zależy mu na dokładnym terminie dostawy. Może on w takim przypadku zlecić dostawę tablicy w formie transportu okazjonalnego, z mniejszymi kosztami transportu. Przewoźnik mógłby zabrać towar i zawieźć go na miejsce docelowe, gdy akurat wykonywałby trasę bez ładunku i kierował się w przybliżonym kierunku. Taka sytuacja jest korzystna dla obu stron, przewoźnik może odbywać przewozy bardziej efektywnie, dzięki nie marnowaniu zasobów na puste przebiegi. Zleceniodawca natomiast, może oczekiwać niższych kosztów przewozu towarów.

Połączenie między osobami zlecającymi usługi transportowe - zleceniodawcami, a osobami oferującymi przewóz towaru - przewoźnikami, może odbywać się za pomocą serwisu oferującego dodawanie publicznych ogłoszeń. Ogłoszenia dzielić się będą na dwie kategorie, ogłoszenie z planowaną trasą przejazdu oraz zlecenie z wymaganym towarem do przewiezienia z punktu początkowego do punktu docelowego.

\section{Cel i zakres projektu}
\label{sec:cele}
Celem projektu jest stworzenie aplikacji webowej, pod tytułem \texttt{CargoLink}. Serwis ten pozwalać będzie na dodawanie ogłoszeń oraz zleceń dotyczących transportów okazjonalnych. Aplikacja przyczni się do zoptymalizowania procesów logistycznych, eliminując nieefektywne wykorzystanie zasobów transportowych. Dodatkowo pozwoli ona przewoźnikom i zleceniodawcom, na łatwą i szybką komunikacje między sobą.
Główne cele projektu:
\begin{enumerate}
    \item \textbf{Możliwość umieszczania ogłoszeń i zleceń}: aplikacja pozwalać ma na dodawanie ogłoszeń o trasach przejazdu, planowanych przez przewoźników oraz zleceń transportowych na konkretny towar przez zleceniodawców
    \item \textbf{Ułatwienie szukania odpowiednich ogłoszeń i zleceń}: poprzez system powiązania zleceń transportowych do planowanych tras przewoźników oraz ogłoszeń o planowanej trasie do zleceń zamieszczonych w serwisie, aplikacja skróci czas potrzebny na znalezienie odpowiednich ofert.
    \item \textbf{Komunikacja między przewoźnikami i zleceniodawcami}: aplikacja umożliwi szybką komunikację między użytkownikami serwisu poprzez czat tekstowy.
    \item \textbf{Redukcja kosztów transportu}: dzięki lepszemu dopasowaniu potencjalnych przewoźników i zleceniodawców, aplikacja pozwoli na obniżenie kosztów transportu zarówno dla zleceniodawców, jak i przewoźników.
\end{enumerate}
Nowoczesna aplikacja transportowa powinna przyczynić się do efektywnego zrealizowania tych celów, co wspomoże rynek zleceń transportowych, przynosząc korzyści zarówno dla zleceniodawców, jak i przewoźników.

\section{Wymagania aplikacji}
Na podstawie analizy celów wymienionych w podrozdziale \ref{sec:cele}, można wywnioskować, że do efektywnego działania serwisu, będą musiały zostać zrealizowane następujące wymagania funkcjonalne:
\begin{enumerate}
    \item \textbf{Uwierzytelnianie}: aplikacja będzie wykorzystywała system rejestracji oraz logowania.
    \item \textbf{Regulanim}: podczas rejestrowania się do serwisu, użytkownik musi zaakceptować regulamin korzystania z aplikacji.
    \item \textbf{Dodawanie zleceń transportowych}: serwis pomagał będzie znaleźć odpowiedniego przewoźnika poprzez udostępnienie możliwości dodania ogłoszenia zlecenia. W ogłoszeniu zlecenia znajdować się będzie:
    \begin{itemize}
        \item tytułu zlecenia,
        \item opis zlecenia (niewymagane),
        \item wymagana trasa przewozu (miejsce startu oraz miejsce docelowe),
        \item przybliżony termin dostarczenia (przedział dat),
        \item waga towarów do przewiezenia,
        \item wymiary przewożonych dóbr,
        \item kategoria każdego z towarów,
        \item informacja o wymaganych specjalnych warunkach podczas transportu (np. jedzenie wymagać będzie przewozu w odpowiedniej temperaturze, pole niewymagane),
        \item imienia i nazwiska zleceniodawcy bądź nazwy firmy, która zleceniodawca reprezentuje,
    \end{itemize}
    \item \textbf{Dodawanie ogłoszeń o planowanej trasie}: system powinien pozwalać przewoźnikom, na dodawanie publicznych informacji o planowanych przez siebie trasach. Ogłoszenie będzie składało się z:
    \begin{itemize}
        \item tytułu ogłoszenia,
        \item opisu ogłoszenia (niewymagane),
        \item planowanej trasy przewozu (miejsce startu oraz miejsce docelowe),
        \item daty planowanego przejazdu,
        \item dostępnego miejsca w pojeździe (wymiary liczone w europaletach),
        \item maksymalnej wagi towaru,
        \item marki i modelu pojazdu,
        \item danych kontaktowych,
        \item imienia i nazwiska przewoźnika bądź nazwy firmy, która przewoźnik reprezentuje,
    \end{itemize}
    \item \textbf{System rekomendacji ogłoszeń}: podczas wprowadzania danych o trasie, użytkownik będzie informowany o sugerowanych zleceniach dodanych przez innych użytkowników (np. gdy zlecenie dotyczy trasy, która w przybliżeniu pokrywa się z tą jaką przewoźnik planuje się poruszać). Analogicznie gdy zleceniodawca zamierza dodać ogłoszenie zlecenia, zostanie on poimformowany o proponowanych ogłoszeniach przewoźników.
    \item \textbf{Umożliwienie konaktu między użytkownikami}: aby zapewnnić kontakt między użytkownikami, w aplikacji zostanie dodany czat tekstowy umożliwiający korespondencje między zleceniodawcami, a przewoźnikami, bezpośrednio w aplikacji. Ma on pełnić rolę komunikacji na wzór tej  oferowanej przez tradycyjną poczte elektroniczną.
    \item \textbf{Generowanie szablonu umowy}: przewoźnik i zleceniodawca, po negocjacji warunków umowy, otrzymają wygenerowany przez serwis szablon dokumentu finalizujący transakcje.
    \item \textbf{Weryfikacja dodawanych ogłoszeń}: zanim ogłoszenie wyświetlać się będzie dla wszystkich użytkowników, moderator serwisu będzię musiał je zatwierdzić.
    \item \textbf{Graficzne przedstawienie trasy}: podczas dodawania ogłoszenia o planowanej trasie, przewoźnikowi wyświetlać się będzie mapa z zaznaczonymi trasami, dodanymi przez zleceniodawców, które przebiegają w przybliżonym kierunku. Zleceniodawcy natomiast, podczas dodawania oferty, będą widzieć mapę ze wszystkimi trasami planowanymi przez przewoźników.
    \item \textbf{System ocen użytkownika}: po 2 dniach po upływie terminu przewozu, użytkownicy będą mogli dodać opinie na temat użytkownika, którego umowa dotyczyła. Opinia składała się będzie z oceny w skali 1-5 oraz komentarza. Opinie będą wyświetlały się na stronie porfilu tego użytkownika oraz na dodanych przez niego ogłoszeniach.
\end{enumerate}
Aplikacja musi być niezawodna i przyjazna do użytkowania dla wszystkich. Do komfortowego korzystania z serwisu przez użytkowników, niezbędna będzie realizacja następujących wymagań niefunkcjonalnych
\begin{enumerate}
    \item \textbf{Innowacyjność}: Aplikacja musi używać nowoczesnych technologii, takich jak \texttt{TypeScript}, \texttt{Next.js}, \texttt{Tailwind CSS}, \texttt{Node.js} oraz bazę danych \texttt{PostgreSQL}, zapewni to wysoką wydajność, skalowalność i bezpieczeństwo aplikacji.
    \item \textbf{Intuicyjny interfejs użytkownika}: Aplikacja musi posiadać prosty i intuicyjny interfejs użytkownika, który umożliwi łatwą obsługę zarówno dla zleceniodawców, jak i przewoźników.
    \item \textbf{Dostępność na różnych urządzeniach}: Aplikacja musi responsywna i dostosowana do różnych urządzeń, takich jak komputery, tablety i smartfony, co zapewni wygodę użytkowania w dowolnym miejscu i czasie.
    \item \textbf{Wielojęzyczność}: użytkownicy korzystający z aplikacji, muszą mieć możliwość wyboru jednego z trzech przewidzianych języków: polski, angielski oraz niemiecki. Dodatkowo użytkownicy podczas rejestracji muszą mieć możliwość wybrania dowolnych języków, którymi się posługują. Informacje te będą zamieszczone na profilu użytkownika.
\end{enumerate}
